% Proudly written in LaTeX by Giulio & Stefano
\title{Errata corrige}
\author{...}
\date{\today}

% Parametri di impaginazione: margini, formato pagina, dimensione e tipo di font
\documentclass[twoside, 12pt]{article}
\usepackage[left=2cm,right=2cm,top=2.5cm,bottom=2.5cm]{geometry}

% Pacchetto per pagine intenzionalmente bianche
\usepackage{emptypage}

% Pacchetti per il font encoding e la lingua del testo
\usepackage[latin1]{inputenc}
\usepackage[T1]{fontenc}
\usepackage[italian]{babel}
% Fonts: testo, formule matematiche e simboli generici
\usepackage{helvet}
\usepackage{courier}
\usepackage[light,math]{iwona}
\usepackage{pifont}
\usepackage{amssymb,amsfonts,textcomp}

\usepackage{amsmath}
\usepackage[makeroom]{cancel}

\usepackage{nicefrac}

% URL e relativa personalizzazione
\usepackage{hyperref}
\hypersetup{pdftex, hidelinks=true,linkcolor=blue, citecolor=blue, filecolor=blue, urlcolor=blue, pdftitle=}

% Table Of Content: mostra solo capitoli e sottocapitoli
\setcounter{tocdepth}{2}

% Pacchetto relativo all'impaginazione a colonne
\usepackage{multicol}

%%% Colori, grafica, figure e tabelle %%%
\usepackage[usenames,dvipsnames]{xcolor}
\definecolor{light-gray}{gray}{0.95}

\usepackage[pdftex]{graphicx}
\usepackage{caption}
\captionsetup{justification=centering}

% Listati e codice
\usepackage{listings}

% Linguaggio Arduino
\lstdefinestyle{Carduino}{
    basicstyle=\footnotesize\ttfamily,
    language=C,
    numbers=left,
    numbersep=5pt,
    numberstyle=\tiny,
    stepnumber=5,
    firstnumber=1,
    numberfirstline=true,
    keywords={void, boolean, unsigned, byte, int, long, short, float, const, String, HIGH, LOW, INPUT, OUTPUT, INPUT_PULLUP},
    %~ keywordstyle=\color[rgb]{.08,.63,.64},
    keywordstyle=\bfseries,
    keywords=[2]{setup, loop, char, word, double, if, else, for, switch, case, while, do, while, break, continue, return},
    %~ keywordstyle=[2]{\color[rgb]{.37,.43,.01}},
    keywordstyle=[2]{\bfseries},
    keywords=[3]{pinMode, digitalWrite, digitalRead, analogReference, analogRead, analogWrite, tone, noTone, shiftOut, shiftIn, pulseIn, millis, micros, delay, delayMicroseconds, min, max, abs, constrain, map, pow, sqrt, sin, cos, tan, randomSeed, random, lowByte, highByte, bitRead, bitWrite, bitSet, bitClear, bit, attachInterrupt, detachInterrupt, interrupts, noInterrupts},
    %~ keywordstyle=[3]{\color[rgb]{.83,.34,.02}},
    keywordstyle=[3]{\bfseries},
    commentstyle=\color[rgb]{.35,.41,.43},
    stringstyle=\color{blue}
}

\lstdefinestyle{bash} {
  language=bash,
}

\lstdefinestyle{nonumbers} {
  numbers=none,
}

\lstdefinestyle{small} {
  basicstyle=\scriptsize\ttfamily,
}

\lstdefinestyle{CarduinoInline} {
  language=C,
  basicstyle=\footnotesize\ttfamily,
}

% wrapping
\lstdefinestyle{wrap} {
    %~ breaklines=true,
    postbreak=\raisebox{0ex}[0ex][0ex]{\ensuremath{\color{red}\hookrightarrow\space}}
}

% impostazioni di default
\lstset {
  %~ frame=single,
  breaklines=true,
  basicstyle=\footnotesize\ttfamily,
  mathescape,
}

%%%%%%%%%%%%%%%%%%%%%%%%%%%%%%%%%%%%%%%%%%%%%%%%%%%%%%%%%%%%%%%%%%%%%%%%%%%%
%%               Direi che qui comincia il documento                      %%
%%                                                                        %%
%%%%%%%%%%%%%%%%%%%%%%%%%%%%%%%%%%%%%%%%%%%%%%%%%%%%%%%%%%%%%%%%%%%%%%%%%%%%

\begin{document}
\maketitle

\section*{Fino a quarta ristampa}

\begin{description}
\item [Pagina V] \textit{``fare del tinkering"}, nella citazione ripresa dal libro di Banzi. Errore di battitura;
\item [Pagina 42] le due formule per calcolare la resistenza del sensore $R_{sens}$ sono errate e devono essere corrette come segue:
\begin{multicols}{2}
\centering
In termini di tensioni:
$$ V_{part} = 5V \cdot \frac {R}{R+R_{sens}} $$
\textcolor{red}{$$ \cancel{R_{sens} = \frac {R}{\frac{5V}{V_{part}}-1}} $$}
$$ R_{sens} = R \cdot \bigg( \frac{5V}{V_{part}}-1 \bigg) $$

Utilizzando le letture di Arduino:
$$ lettura = 1024 \cdot \frac {R}{R+R_{sens}} $$
\textcolor{red}{$$ \cancel{R_{sens} = \frac {R}{\frac{1024}{lettura}-1}} $$}
$$ R_{sens} = R \cdot \bigg( \frac{1024}{lettura}-1 \bigg) $$
\end{multicols}

Di conseguenza, i codici che implementano queste formule devono essere corretti inserendo una moltiplicazione al posto di una divisione. In particolare, a \textbf{Pagina 43} Linea 15 e \textbf{Pagina 46} Linea 19:
\begin{lstlisting}[style=Carduino,style=wrap]
float resistenzaSensore = RESISTENZAFISSA * ( 1024.0 / valoreSensore - 1 );
\end{lstlisting}

\item[Pagina 46] Refuso nel codice: l'identificatore del pin A0 � \texttt{PIN\_TERMORESISTENZA} (e non\\
\texttt{PIN\_FOTORESISTENZA} come nel listato precedente).
\begin{lstlisting}[style=Carduino,style=wrap,firstnumber=7]
pinMode(PIN_TERMORESISTENZA, INPUT);
\end{lstlisting}
\begin{lstlisting}[style=Carduino,style=wrap,firstnumber=14]
int valoreSensore = analogRead(PIN_TERMORESISTENZA);
\end{lstlisting}
\end{description}

\section*{Fino a seconda ristampa}
\begin{description}
\item[Pagina 18] ``6 pin, \textit{due} per ogni colore";

\item[Pagina 37] � presente una parentesi tonda in pi� alla linea 11:
\begin{lstlisting}[style=Carduino,style=wrap,firstnumber=11]
float tensione = lettura * 5.0 / 1024.0;
\end{lstlisting}

\item [Pagina 38] precisazione: si usa \textbf{float} non solo perch� la tensione � pi� piccola di 1V ma perch� in generale � decimale;

\item[Pagina 47] c'� un cambio di segno alla riga 33, come evidenziato nel commento corretto:
\begin{lstlisting}[style=Carduino,style=wrap,firstnumber=33]
temperatura = 1/temperatura - 273.15;
\end{lstlisting}

\item[Pagina 53] precisazione: \textbf{``la durata di un singolo impulso � detto periodo"}. Il periodo � la somma dei tratti in cui la tensione � \texttt{HIGH} ed in cui � \texttt{LOW};

\end{description}

\end{document}
